This project uses the open source project ``OpenReception'' as case study. The project aims to provide a drop-in replacement for an existing system, and therefore has relatively fixed requirements that are extracted from the workings of the existing system. The system is developed and released under an open source license and any implementation details are therefore public domain and not covered by any non-disclosure agreements. This section gives a short introduction to the system, its architecture and design and the development process that ultimately motivated the test approach, that is the topic of this thesis.\\\\
OpenReception web-based software/telephony system. It is a system designed to enable receptionists to handle incoming calls, and provide then with the appropriate information so that they may divert or directly handle the calls. The system is designed with high availability in mind with many -- largely independent -- components that are loosely coupled. This limits the Domino-effect, where one faulty component can take down another for no other reason than the fact that they are partitioned together.\\ This component-oriented design has also helped the testing process, as it enabled individual components to be tested and verified independently of the others.

\section{Project scope}
The fundemental requirements for the system originates directly from the fact, that is is supposed to be a drop-in replacement of an exisiting system. It should therefore, as a bare minimum, mirror the features of the existing system.\\
However, the current system has been in production for over ten years and lessons-learned has taught the customer how the system should be improved. %TODO maybe add a list of requirements.

%Use cases are stored in a wiki.
