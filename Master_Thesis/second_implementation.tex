\subsection{Framework content}

During the writing and refinement of the testing model defined in 
%Service layer, model layer and more.


%TODO move this section to a main section
\section{Adding domain knowledge}
%STUB
If we were to add additional domain knowledge to the use cases, then we would also be able to extract capabilities easily.
%yada yada, use case example: actor does something to some other thing
% it means the the actor can do something to some other thing.

Taking one step up, trying to abstract away the current problem domain, what problem did we actually try to solve? We wanted to verify expected behavior of a system that we treat as being a black-box system. We use the holy grail of behavior -- the use cases -- as reference point to how the system should behave in difference situations.

%NOTE: use cases, and requirements must be expected to be incomplete (find some historical reference), and could possibly be thought into the development process,. making the use cases and requirement a more living part of the development.

%How to structure and formalize/represent requirements
Given the high rate of software project failures, and the general widespread requirement/implementation mismatch, it would be safe to claim that requirements (and use cases) are incomplete with regards to system working. However, during the development life-cycle, additional domain knowledge is bound be acquired. This knowledge, other improving then general understanding of the problem domain, may also affect the the requirements by either; making them more elaborate and/or complete. For example, a boundary case of a use case may be appear clearer (TODO: EXAMPLE). But, the acquired knowledge may also lead to requirement changes, either due to decreased implementation complexity, or simply non-realizable requirement. (TODO: EXAMPLE)

In any case, assuming requirements evolve and shift during development. We do nice stuff.
%TODO STUB.

\section{Development process}
\begin{figure}[ht]
\centering
\includegraphics[width=0.7\textwidth]{\imgdir ideal_flow}
\caption{Ideal development flow}
\label{fig:ideal_flow}
\end{figure}

\begin{figure}[ht]
\centering
\includegraphics[width=0.7\textwidth]{\imgdir event-stack-to-state-machine}
\caption{Concept; validate event stack using life-cycle state machines.}
\label{fig:event-stack-to-state-machine}
\end{figure}

\begin{figure}[ht]
\centering
\begin{drawstack}
  % Within the environment, draw stack elements with \cell{...}
  \cell{lock}
  \cell{unlock}
\end{drawstack}
\caption{Event stack}
\label{fig:event-stack-example}
\end{figure}
