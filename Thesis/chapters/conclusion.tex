\chapter{Conclusion}
The road to an (realistically) implementable product has been long and filled with dead-ends.\medskip

\noindent In conclusion; the goal of having our tool users write use cases -- as close to use cases as possible, is not feasible in regards to automatic test generation without enforcing restrictions on use case writing. As this thesis focused in enforcing as little restrictions on use case writing as possible -- the solution proposed here is faulty by design, as it contradicts itself.\medskip

\noindent While the tool implemented in the context of thesis may not be feasible as a general methodology, the artifacts that it produced has helped the case study system in two major ways: The use case driven development approach aided in getting the system stabilized, and aided to focus on functionality, rather than features. The other aid, was the test support tools that provided an excellent component-based structure for writing additional tests. The actor and concept classes, along with the service-oriented architecture made the test very readable, and even implicitly be read as requirements.
\section{Further work}
\begin{itemize}
  \item Integrate into version control system
  \item Integrate into wiki system
  \item Validating event chain against a state machine (automaton).
  \item Executable activity diagrams - knieke2010 
  \item The use case-as-precondtions could be implemented via a carried environment.. (explained somewhat in design).
\end{itemize}

%\subsection{Interpretor-like use case steps}

\subsection{UI testing}
In the last few months use case-focused tests have been written for the OpenReception ReceptionistClient application using the Selenium project\footnote{http://www.seleniumhq.org/}.

\begin{itemize}
  \item A common programming language for tests and main code base matters.
\end{itemize}

\section{Findings}

%Continuous integration services are increasing in popularity -- and for a reason. It makes testing a more intricate part of the development, and the approach presented in this thesis takes it a whole new level.
