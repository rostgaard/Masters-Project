\section{Concept 1 - Markdown editor}
%Same concept as a ritch-text editor. Select and make bold.
%Basically put everything into a textual format.
\begin{figure}[!htbp]
\centering
\includegraphics[width=0.80\textwidth]{\imgdir markdown_ui_mockup}
\caption{Early mock-up of a user interface using a markdown-like language for writing use cases. Left part of the UI shows a structured textual representation of a use case. The middle part was reserved for graphical representations of the use case, which could be sequence diagrams or activity diagrams}
\label{fig:markdown_ui_mockup}
\end{figure}
An early idea was to provide a markup language with the possibility to tag specific words as keywords, which then became the concepts that were included in the...
This concept was only implemented as an experimental prototype model without user interface, and proved too loose in structure to be able to implement within the time frame of this Thesis. The main problem was that it required a domain-specific language to be able to support the structure needed for test generation.

%The challenge in documenting requirements is, and has always been, to formulate them on a non-ambiguous form. The quantification of ambiguity tends to be difficult as well, due to the fact that requirements are usually formulated in natural languages following some rules, such as pre-defined glossary and constraints. Constraints vocabulary typically consists of should, could, may, must.

%Other method
Wiki-collaboration to "lazily" define the problem domain and verification conditions along the way.

The concept of clear-text analysis lived on.

