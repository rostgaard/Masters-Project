\chapter{Implementation}
This chapter describe the tool that was implemented as part of this thesis, to enable the generation of tests from use cases. The chapter provides a high-level architectural view of the test system, and an example on how a use case goes from description to test.\medskip

\noindent As already mentioned in chapter \ref{ch:design}, the actions of users were eventually dropped, as they provided little or no value in test. This will become more evident, when an example of a generated test is shown.\medskip

\noindent The protocol specification for the implementation is shown in appendix \ref{appendix:protocol}, test results for it can be found in appendix \ref{appendix:tests}, and the database schema is located in appendix \ref{appendix:database-schema}. The handbook, written from the point of view of the user interface, can be found in appendix \ref{appendix:handbook}.

\section{Architecture}
The implementation is done in a client/server architecture. It consists of a three basic components; a client, a service and a shared library. These map to the folder structure. It is web based to allow easy collaboration.\medskip
\begin{figure}[!htbp]
  \centering
  \includegraphics[scale=0.7]{\imgdir layered-mvc}
  \caption{Layered model-view-control architecture with test backend}
  \label{fig:layered-mvc}
\end{figure}
\noindent The distributed model is built around a layered model-view-control (MVC), that uses a dedicated test backend. Our backend built upon the Dart unittest library configured for junit xml output, the Jenkins continuous integration server, and Git revision control system. Git is used, first and foremost, to control revisions, but also as a test code distribution system. The Jenkins server will se a change in the Git repository, update to this version, and trigger a new build. Being that the tests output junit-format xml documents, Jenkins can parse this and provide detailed reports on success/failure output, run-times, and historical data.\medskip

\noindent The layered MVC introduced above, is a model commonly used in web programming. It is illustrated in figure \ref{fig:layered-mvc}, and is a method for maintaining the MVC paradigm, in a distributed system. In it's simplicity, it has a service interface that provide an API. This API delegates to an internal controller, that is allowed to modify the model. This model is only stored on by the service, but may be retrieved (or even pushed) to a client's model via a representation, which then becomes the service's view. The client's model is pushed to a local view, which is a user interface, that may send change requests to its local controller that delegates it to the controller of the service.\medskip

\noindent The client and the server share their code-base around a library package that contain (meta model) models, serialization/de-serialization tools, and the (use case) translation tools. The translation can, thus be done either by the client, or the server.

\section{Use case translation}
This section contains the a description of the different implemented components and utilities needed to support the use case to test translation, and details of the use case details as well.\medskip

\noindent A use case that is unfolded to a set of paths, each having its own test function, will produce a log transcript on every run. This output log includes a list of assumptions (see section \ref{ssec:simulating-error-conditions}), and potentially an error. Errors arise when an expectation failed, or a system error occurs and are fatal for the test run -- i.e. the run will at the point where the error occurred. Assumptions are for use case entries that are mapped to assumption statements, and are non-fatal.

\subsection{Test templates}
Test templates are, in this thesis, hand-written source code files that are missing the actual test code function bodies. They provide the tests with the implemented functions of what they need. So if a test makes a function call to a function called ``receptionist\_answers\_call(receptionist, call)'' with the passed parameters ``receptionist'' and ``call'', then the template is expected to provide this function.\medskip

\noindent The templates are implemented in a simple form. They are source code files that have a single placeholder entry that is replaced with the generated code by the tool. The placeholder is located within a comment, so template files can still be parsed and complied before the generation step, enabling them to be verified prior to use.\medskip
The templates are the mapping link between use case and the system under development, and are -- for the example use in this chapter -- linked to our case study system (section \ref{sec:case-study-system}.

\subsection{Actor and concept classes}
Actor classes represent a domain actor, and should be able support the behavior that is expected from an actor of that type. The classes will be hand-written by a developer, but the functionality that is needed, may be inferred by the use case steps involving the actor/concept and the actions it performs.

\subsection{Definition}
Definitions are mappings from use case roles -- which are simple string identifiers -- to actor, or concept types. Definitions are stored in a global definition set that may be reused by different use cases. The definitions are used to determine which actor and concept types that should be part of the signature of; both the individual use case steps, but also the signature of the use case path functions.

\subsection{Normalization of use case steps}
To be able to make a use case step into a function call, we need to normalize it to a string that follows the general conventions of function call in programming languages (and specific for Dart, as this is our target language). We also need to be able to recognize identical steps in different use cases. It gives us the following constraints for a normalization function.
\begin{itemize}
  \item Programming conventions must be followed. This means no spaces in function identifier, no redeclaration of function identifiers with different parameter set, and no identifiers must start with a number.
  \item Function should be deterministic. So given the an specific input, it always produces the same output.
  \item Not specifically a constraint; but a the function should -- if at all possible -- produce human-readable output to make it easier to trace generated code.
\end{itemize}
The normalization described above, could be realized by a good hashing algorithm -- prefixed by a constant string to avoid prefixed numbers in identifiers. This, however would collide with our desire to have human-readable functions. The implementation ended up doing the following:
\begin{itemize}
  \item Replace all non-allowed characters with underscores.
  \item Prefix the functions with an underscore
  \item Transform the every character in the string of step to lower case
\end{itemize}
A function identifier for ``Receptionist answers call'' then becomes:
\begin{verbatim}
  _receptionist_answers_call
\end{verbatim}
One thing that is needed for a function \emph{call}, are the function parameters. These are supplied by the definitions associated with the use case. So, given that the set of definitions contain the ``receptionist'' and ``call', the signature of the function would look like this:
\begin{verbatim}
  _receptionist_answers_call (receptionist, call)
\end{verbatim}

\section{Example translation}
This section goes through the steps associated with creating tests from use cases in this implementation. The example is done, disregarding the concrete user interface details, which can instead be found in the handbook in appendix \ref{appendix:handbook}.\medskip

\noindent The example reuses the use case from the 3rd concept section (\ref{sec:3rd-iteration}), that is also repeated in figure \ref{lst:uc-simple-example-revisited} for convenience.

\begin{lstlisting}[frame=single,style=usecase, caption=Use case example revisited, label=lst:uc-simple-example-revisited]
Scenario:
  Receptionist types in message
  Receptionist sends message
  Receptionist marks state as ready 
Preconditions:
  The receptionist is created
  The receptionist is logged in
Postconditions:
  The message is stored
  The receptionist is ready to handle the next call
\end{lstlisting}
As mentioned earlier, the modeling of an ``action'' was dropped because it provided no real value in test generation. With this ignored, the highlighting of actors (orange) and concepts (blue) -- done by the use case writer -- will look like listing \ref{lst:uc-simple-example-highlighted-revisited}, and lead to two declarations; one declaration of the role of ``receptionist'' actor of type ``receptionist'', and a domain concept role of a ``message'' of type ``message''. The concrete user interface will have containers that signify where pre- and postconditions and main scenario, effectively giving us the composition and structure of the use case.

\begin{lstlisting}[frame=single,style=usecase, caption=Use case example with its different parts highlighted, label=lst:uc-simple-example-highlighted-revisited]
Scenario:
  @\color{orange} Receptionist@ types in @\color{blue}{message}@
  @\color{orange} Receptionist@ sends @\color{blue}message@
  @\color{orange} Receptionist@ marks state as ready
Preconditions:
  The @\color{orange}receptionist@ is created
  The @\color{orange}receptionist@ is logged in
Postconditions:
  The @\color{blue}message@ is stored
  The @\color{orange}receptionist@ is ready to handle the next call
\end{lstlisting} 
When the declarations are in place, we can pass the structured use case, along with them to the generator. It will perform normalization on the steps (and actor roles), in order to build a test that does not break the syntax of the programming language. Listing \ref{lst:generated-test-code-example} show the generated code with normalizations applied. A thing to note, is the function signature for the generated functions includes every concept and actor that is used within their bodies. This is done to support chaining model presented back in section \ref{ssec:step-execution}. The code generated here is clearly not runnable, as the called functions is not declared. We need to write up a template that we can place the functions in.
\begin{lstlisting}[style=Dart, caption=Example of generated code without a template applied concept,label={lst:generated-test-code-example}]

void preconditions(Receptionist receptionist) {
  the_receptionist_is_created(receptionist);
  the_receptionist_is_logged_in(receptionist);
}

void postconditions(Receptionist receptionist, Message message) {
  the_message_is_stored(message);
  the_receptionist_is_ready_to_handle_the_next_call(receptionist);
}

void scenario(Receptionist receptionist, Message message) {
  preconditions(receptionist, message);
  receptionist_types_in_message(receptionist, message);
  receptionist_sends_message(receptionist, message);
  receptionist_marks_state_as_ready (receptionist)  
  postconditions(receptionist, message);
}

\end{lstlisting}

\noindent Listing \ref{lst:example-template-methods} show the template that supports the generated use case from listing \ref{lst:generated-test-code-example}. 

The unmet condition is handled here by writing a code block that throws an \texttt{AssertionFailure}. 
\begin{lstlisting}[style=Dart, caption=Example template methods (written manually),label={lst:example-template-methods}]
UserService userService = new UserService(...);
MessageService messageService = new MessageService(...);
DummyMessageFactory messageFactory = new DummyMessageFactory(...);
  
void the_receptionist_is_created(Receptionist receptionist) {
  try (userService.get(receptionist.user.id) {}
  catch (NotFoundException e) {
    throw new AssertionFailure("Receptionist " + 
                                receptionist + 
                               " not created");
  }
}

/* Later steps will fail if the receptionist is not logged in. */
void the_receptionist_is_logged_in(receptionist) {
   assume("the_receptionist_is_logged_in");
}

void the_message_is_stored(Message message) {
  Message fetchedMessage = messagService.get(message.id);
  expect (fetchedMessage, equals (message));
}

void the_receptionist_is_ready_to_handle_the_next_call(Receptionist r) {
  expect (r.state, equals(ReceptionistState.Idle);
}

void postconditions(Receptionist receptionist, Message message) {
  the_message_is_stored(message)
  the_receptionist_is_ready_to_handle_the_next_call(receptionist);
}

void receptionist_types_in_message(Receptionist r, Message message) {
  // Generate() takes a Receptionist parameter - signifies author.
  message = messageFactory.generate(r);
}

void receptionist_sends_message(receptionist, message) {
  messageService.send(message);
}

void receptionist_marks_state_as_ready(receptionist) {
  userService.changeState(receptionist.user);
} 

/*[USE-CASE-PLACEHOLDER]*/
\end{lstlisting}

%TODO Verify that this is needed.
%\section{The mapping process}
%Some basics on the procedure; take every line of the use case and not every concept and actor used. Then speculate on the realization of this. Which components should be involved, and which other actors. Depending on the concrete architecture, these components may be services, larger program components (such as Java packages), or even sub-functions.

\subsection{Running the analysis}
Having a set of definitions, we can detect actors and concepts from textually analysing the text of every UseCaseEntry object. The analysis is quite simple, and merely looks for occurrences of the definition by comparing strings.

\subsection{Initial state}
The initial state of the test support tools are provided by a global setup and teardown function that encapsulates the entire pool of tests.


\section{Summary}
The implementation is on a prototype stage and contains some open ends.

The use case-as-precondtions could be implemented via a carried environment.. (explained somewhat in design).

In the test support tools written for the test case system, we make heavy use of resource pools and object factories to supply the tests with the, quite large, resource blocks that they need.
