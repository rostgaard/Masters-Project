\chapter{Implementation}
%lead-in
The implementation is done in a client/server architecture. It consists of a three basic components; a client, a service and a shared library. These map to the folder structure. It is web based to allow easy collaboration.

%TODO add MVC to glossary.
Everything is built up as a layered model-view-control (MVC), that share a common library of models, utilities and interfaces.
%TODO what does the layer mean?

%identities of objects.

\subsection{Client}

\section{Example translation}
%TODO enter the use case.

\begin{lstlisting}[frame=single,style=usecase, caption=Use case example revisited, label=lst:uc-simple-example-revisited]
Scenario:
  Receptionist types in message
  Receptionist sends message
  Receptionist marks state as ready 
Preconditions:
  The receptionist is created
  The receptionist is logged in
Postconditions:
  The message is stored
  The receptionist is ready to handle the next call
\end{lstlisting}
%TODO verify that we have dropped the "action" concept in design concept 3.

\begin{lstlisting}[frame=single,style=usecase, caption=Use case example with its different parts highlighted, label=lst:uc-simple-example-highlighted-revisited]
Scenario:
  @\color{orange} Receptionist@ types in @\color{blue}{message}@
  @\color{orange} Receptionist@ sends @\color{blue}message@
  @\color{orange} Receptionist@ marks state as ready
Preconditions:
  The @\color{orange}receptionist@ is created
  The @\color{orange}receptionist@ is logged in
Postconditions:
  The @\color{blue}message@ is stored
  The @\color{orange}receptionist@ is ready to handle the next call
\end{lstlisting} 

\begin{lstlisting}[style=Dart, caption=Example of generated code without a template applied concept,label={lst:generated-test-code-example}]

void preconditions(Receptionist receptionist) {
  the_receptionist_is_created(receptionist);
  the_receptionist_is_logged_in(receptionist);
}

void postconditions(Receptionist receptionist, Message message) {
  the_message_is_stored(message);
  the_receptionist_is_ready_to_handle_the_next_call(receptionist);
}

void scenario(Receptionist receptionist, Message message) {
  preconditions(receptionist, message);
  receptionist_types_in_message(receptionist, message);
  receptionist_sends_message(receptionist, message);
  receptionist_marks_state_as_ready (receptionist)  
  postconditions(receptionist, message);
}

\end{lstlisting}


\begin{lstlisting}[style=Dart, caption=Example template methods (written manually),label={lst:generated-test-code-example}]
UserService userService = new UserService(...);
MessageService messageService = new MessageService(...);
DummyMessageFactory messageFactory = new DummyMessageFactory(...);
  
void the_receptionist_is_created(Receptionist receptionist) {
  try (userService.get(receptionist.user.id) {}
  catch (NotFoundException e) {
    throw new AssertionFailure("Receptionist " + 
                                receptionist + 
                               " not created");
  }
}

/* Later steps will fail if the receptionist is not logged in. */
void the_receptionist_is_logged_in(receptionist) {
   assume("the_receptionist_is_logged_in");
}

void the_message_is_stored(Message message) {
  Message fetchedMessage = messagService.get(message.id);
  expect (fetchedMessage, equals (message));
}

void the_receptionist_is_ready_to_handle_the_next_call(Receptionist r) {
  expect (r.state, equals(ReceptionistState.Idle);
}

void postconditions(Receptionist receptionist, Message message) {
  the_message_is_stored(message)
  the_receptionist_is_ready_to_handle_the_next_call(receptionist);
}

void receptionist_types_in_message(Receptionist r, Message message) {
  // Generate() takes a Receptionist parameter - signifies author.
  message = messageFactory.generate(r);
}

void receptionist_sends_message(receptionist, message) {
  messageService.send(message);
}

void receptionist_marks_state_as_ready(receptionist) {
  userService.changeState(receptionist.user);
} 

\end{lstlisting}