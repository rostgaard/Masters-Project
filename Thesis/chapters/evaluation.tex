\chapter{Evaluation and discussion}

\section{Parallelism}
A section on multiple actors and race conditions. How do you create a use case that contains multiple simultaneous instances of actors that perform the same action synchronously? Basically asserting parallel properties.\todo{.\\turn this into a real section}
%\section{Applicablity to different methodologies} See http://en.wikipedia.org/wiki/Requirements_analysis
% Waterfall, Prototype model, Incremental, Iterative, V-Model, Spiral, Scrum, Cleanroom, RAD ...

\subsection{Side benefits}
%One of the major benefits is that it makes other black-box tests easier, as we can re-use components from the framework.

% Further work
%  - Misuse case
%  - Validating event chain against a state machine (automaton).
%  - Executable activity diagrams - knieke2010 
% Treating the system global state enforces a notion within the programmer how the specific activity acutally mutates the global state of the system.
Writing tests is generally easier. Especially tests that asserts that an edge case is not reached. The code becomes 

\begin{figure}[!hbpt]
\centering
\includegraphics[width=1.0\textwidth]{\imgdir jenkins-build-trend-interation-1}
\caption{Iteration 1 build trend}
\label{fig:jenkins-build-trend-interation-1}
\end{figure}


\begin{figure}[!hbpt]
\centering
\includegraphics[width=1.0\textwidth]{\imgdir jenkins-build-trend-interation-2}
\caption{Iteration 2 build trend}
\label{fig:jenkins-build-trend-interation-2}
\end{figure}


\section{Feasibility}
Parameters that should be measured; LOC ratio; test tools; Phonio is around 2kLOC and neglecatable

Test framework is \~8.5kLOC and 1.1 of these lines are code that wraps the framework. The ratio is then roughly around 10\% written code lines.

Model framework is also around 8kLOC and is reused in both the client and the server.
\subsection{Relationship to TDD}
In practice; the approach presented in this thesis is a variant of TDD, but injects some elements from MDE and increases the test coverage by automatically generating the tests from use case branches.


\subsection{Is if feasible}
Basically; no. Can it be used? To some extent. The process is nice, but the mapping simply unfolds into a too large state space with a stucture so complex that is impossible for an end-user to specify it, and very hard for a trained professional to elaborate it. It may be feasible once model-based software engineering reaches a level where behavioral model can be applied to domain models, and thus ...

\section{Related work}