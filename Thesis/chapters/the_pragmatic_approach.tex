\chapter{First iterations}
%Purpose is to provide anecdotal background for the theoretical descussion.
%First implementation focused on the generation, second focused on the domain.
This section describes the first and second iteration of the test generation tool. The first iteration was developed using code generation as an experiment that proved itself useful enough to refine. The second refinement was done as part of this thesis, and was linked directly to the shared domain model and interfaces of the system that was exposed in common framework (section \ref{sec:openreception_framework}).\\
This section covers the motivation for building the test, and test tools as we did, and how it was implemented.
\section{First implementation}
During the development of the system, little or no unit tests had prevailed. This  was primarily caused by the very high fluctuation in application architecture, design and implementation. None of the developers were familiar with the development of telephony applications, and lessons-learned frequently affected both design and implementation and quickly made whatever unit tests that was created, obsolescent.\\
The high level of asynchronism of the system, and low level of control with the PBX made feature regressions a daily pain. Unit tests were of no use, as what we needed to test were the components. A common way of testing components is to provide a complete test environment (sometimes called ``test harness'') that emulates the behavior of the interfaces that the component under test requires. In this case, the option of emulating the interface of an entire PBX wasn't really solution for our problem. Basically, if we knew the exact behavior of the PBX, then we would just build the logic to handle it. The best solution we could think of, was to include the PBX as a required -- and included -- component for the component tests.\\\\
As development progressed, it became increasingly difficult and time consuming to verify the correctness of the implementation by manually running the acceptance tests, which basically involved; performing an incoming call via a phone, picking up the call via the system, accept the call on another phone and then perform the actual use case scenario. This could, for example, be to forward the call and signal an idle state to system. Other issues with this manual approach was that is was time-consuming and was easy to perform in an incorrect order, thus leading to false negatives in test runs. Clearly the project could benefit from investing time in scripting the setup and tearing down of the state which was need to perform the manual tests.\\\\
The next challenge was then to determine which actions should be automated, and how much they should be automated? This was easily solved by reverse engineering a partial activity diagram from the use cases, and add the missing parts with the help of a representative from Responsum. Once we had the activity diagram, we decided that every macro-action we wanted to automate (in executable code) mirrored an action in the activity diagram. So, basically, if we had an action that was ``Receptionist dials extension of contact'', we would have a corresponding re-usable function that performed this action. We dubbed this ``use-case oriented scripting''.\\\\
In this scripting environment, we started to outline some of the domain actors (such as contact and receptionists), the services interfaces, and domain concepts (such as calls and messages).

\section{Mapping}

\begin{figure}[ht]
\centering
\includegraphics[width=0.90\textwidth]{\imgdir example_test_case}
\caption{Patterns in generated test case}
\label{fig:domain_model}
\end{figure}
The first implementation merely mapped every action, precondition and postcondition line to a code block. So, for instance, the %TODO example.

Then we assigned every line a unique identification and mapped the identification to individual code chunks meant for three specific purposes;
\begin{description}
  \item[Visualization:] A graphical presentation of the use case in the form of a sequence diagram. In our specific case, we used a tool called seqdiag, and an example of the input is shown is shown in listing \ref{lst:seqdiag_code_example}. The main concept of this was to provide convenient overview of how, and when, the interaction between the actors happened.
  \item[Documentation:] In the OpenReception project, every bit of the documentation lives in a Wiki which is formatted using Markdown. This meant that we could link the source documents associated with use case to the use cases themselves. Generation of use case documentation in the Wiki was using these document fragments.
  \item[Testing] The final, and most significant part of the system, was the ability to generate use case tests from the actual use cases. This meant that we needed to describe every action in the use case as bit of code and then patch it together using the list of actions, pre- and postconditions provided in the use case descriptions. An example of an assembled is shown in listing \ref{lst:example_python_output}.
\end{description}
\begin{lstlisting}[language=Bash, caption=Example seqdiag input, label=lst:seqdiag_code_example]

Receptionist-N ->> Klient-N [label = "genvej: fokus-modtagerliste", note = "maaske"];
Receptionist-N ->> Klient-N [label = "retter modtagerlisten"];
\end{lstlisting}

\begin{lstlisting}[language=Python, caption=Example Python code output, label=lst:example_python_output]
# \${WIKI_URL}
from forward_call import Test_Case
class Sequence_Diagram (Test_Case):
def test_Run (self):
try:
Incoming_Call_ID = self.Preconditions ()
self.Step (Message = "Receptionist-N ->> Klient-N [genvej: ring-til-primaert-nummer]")
Outgoing_Call_ID = self.Receptionist_Places_Call (Number = self.Callee.sip_uri ())
self.Step (Message = "Call-Flow-Control ->> FreeSWITCH [ring-op: telefon-N, nummer]")
self.Callee_Receives_Call ()
self.Step (Message = "FreeSWITCH ->> FreeSWITCH [forbind opkald og telefon-N]")
\end{lstlisting}

\noindent
The observant reader will probably already have noticed that this code cannot stand by itself. For instance, the \texttt{self.Callee\_Receives\_Call()} statement references a method found within its own class\footnote{For those unfamiliar with Python, \texttt{self} is a reference to the current object instance, similar to the \texttt{this} keyword in Java.}. In order to keep the code chunk complexity low, we decided to abstract a lot of the complexity into macro functions that were provided to the use case through a framework. Each use case was given it's own programming class that then provided these macro function via class members. Each class would then need to be self-contained and also provide setup/teardown functions, and setup pre and postconditions. The setup and teardown functions were defined as begin technical prerequisites, whereas the pre- and postcondition functions were for setting up the use case prerequisites and would, thus, map to the use case pre- and postconditions.\\\\
Programing-wise, this was done by adding the needed macro methods required by the use cases to an abstract superclass that represented the overall use case. Each path, or variant, would then be an extension of this superclass. A simplified class diagram illustrating this is shown in figure \ref{fig:first_generation_class_structure}. Missing methods, or problems in generated code were identified by static analysis using the pylint tool\footnote{http://www.pylint.org/}.\todo{.\\example here}\\\\


\begin{figure}[!h]
\centering
\includegraphics[width=0.20\textwidth]{\imgdir first_generation_class_structure}
\caption{Class diagram outlining the overall hierarchy of executable use cases (\textbf{TODO:} Should implement run, pre- and postconditions)}
\label{fig:first_generation_class_structure}
\end{figure}

\begin{figure}
\centering
\includegraphics[width=0.80\textwidth]{\imgdir markdown_ui_mockup}
\caption{Early mock-up of a user interface using a markdown-like language for writing use cases.}
\label{fig:markdown_ui_mockup}
\end{figure}

We soon realized that doing individual mapping for every line was tedious, time consuming and error-prone. Especially due to the fact that a lot of implicit knowledge was needed to perform the manual mapping. An example of this would be to reference a person object in an a scenario action, assuming that it was previously declared in an earlier macro.\\\\
We realized that if we injected additional domain knowledge into the test framework, would reduce the number of errors associated in mapping the tests manually. Specifically, by moving our macro methods to the appropriate class -- in an object-oriented fashion -- would enable reuse. For example when a sentence states that a receptionist actor hangs up a call, we add a ``hangup'' method to the  ``Receptionist'' class that takes a ``Call'' object as an argument.\\\\
An early idea was to provide a markup language with the possibility to tag specific words as keywords, which then became the concepts that were included in the 

%The task felt ripe for automation. But first, we needed some sort of abstract representation of the use cases and thus the idea of use cases as tests were born.

%\section{The verification problem}
%TODO V-model.
\section{Second implementation}
During the evolution of the software, platform, language and architecture changes eventually landed us in a space where we had the opportunity to use the same programming language both on the sever and the client. We wanted to exploit this initially by sharing the model classes between the server and client, but it soon evolved into a larger framework, also covering interfaces and REST resource definitions. This section explains the processes of exploiting this framework for a third application; namely testing.

\subsection{Framework content}

During the writing and refinement of the testing model defined in 
%Service layer, model layer and more.

\section{Resource pools}
%TODO stub.
Tests require resources that could be acquired from pools. Pools are resource containers with a fixed discrete amount of resources in them, or in some cases, containers with infinite amount of resources in them. Either they should hide behind macros, or provide components for other resources.

%TODO move this section to a main section
\section{Adding domain knowledge}
%STUB
If we were to add additional domain knowledge to the use cases, then we would also be able to extract capabilities easily.
%yada yada, use case example: actor does something to some other thing
% it means the the actor can do something to some other thing.

Taking one step up, trying to abstract away the current problem domain, what problem did we actually try to solve? We wanted to verify expected behavior of a system that we treat as being a black-box system. We use the holy grail of behavior -- the use cases -- as reference point to how the system should behave in difference situations.

%NOTE: use cases, and requirements must be expected to be incomplete (find some historical reference), and could possibly be thought into the development process,. making the use cases and requirement a more living part of the development.

%How to structure and formalize/represent requirements
Given the high rate of software project failures, and the general widespread requirement/implementation mismatch, it would be safe to claim that requirements (and use cases) are incomplete with regards to system working. However, during the development life-cycle, additional domain knowledge is bound be acquired. This knowledge, other improving then general understanding of the problem domain, may also affect the the requirements by either; making them more elaborate and/or complete. For example, a boundary case of a use case may be appear clearer (TODO: EXAMPLE). But, the acquired knowledge may also lead to requirement changes, either due to decreased implementation complexity, or simply non-realizable requirement. (TODO: EXAMPLE)

In any case, assuming requirements evolve and shift during development. We do nice stuff.
%TODO STUB.

\section{Development process}
\begin{figure}[ht]
\centering
\includegraphics[width=0.7\textwidth]{\imgdir ideal_flow}
\caption{Ideal development flow}
\label{fig:ideal_flow}
\end{figure}

\begin{figure}[ht]
\centering
\includegraphics[width=0.7\textwidth]{\imgdir event-stack-to-state-machine}
\caption{Concept; validate event stack using life-cycle state machines.}
\label{fig:event-stack-to-state-machine}
\end{figure}

\begin{figure}[ht]
\centering
\begin{drawstack}
  % Within the environment, draw stack elements with \cell{...}
  \cell{lock}
  \cell{unlock}
\end{drawstack}
\caption{Event stack}
\label{fig:event-stack-example}
\end{figure}

Some basics on the procedure; take every line of the use case and not every concept and actor used. Then speculate on the realization of this. Which components should be involved, and which other actors. Depending on the concrete architecture, these components may be services, larger program components (such as Java packages), or even sub-functions.

\begin{figure}
\centering
\includegraphics[width=0.80\textwidth]{\imgdir component_diagram_with_tests}
\caption{Component diagram -- extended with tests}
\label{fig:component_diagram_with_tests}
\end{figure}

