%!TEX root = ../Thesis.tex
\chapter{Glossary}
\label{appendix:glossary}
This appendix contains a list of terms introduced or used in this thesis.
\begin{description}
  \item[Test support framework:] A hand-coded set of source code files that provides interfacing between the sources of application under development, and the tests files that are generated. In this thesis, it builds upon an existing test library designed for unit testing. It provides the needed ``setup'', ``teardown'', grouping and tests runners that catch and report unhandled exceptions.
  \item[CRUD:] Abbreviation of \textbf{C}reate, \textbf{R}ead \textbf{U}pdate and \textbf{D}elete is an acronym denoting the four primitives operation of persistent data storage. An interface for persistent storage may expose all or a subset of these primitives.
  \item[Requirement analysis:] Requirement analysis, for the purpose of this project, meant as a concept that uses some the information stored in the requirements to check it for validity, ambiguity and provide alternative representations. An example of an alternative representation is that a set of use cases becomes a use a use case diagram, just on a different abstraction level.
\end{description}

\chapter{Misfits}

\section{Event stack validation concept}
\label{sec:event-stack-validation}
A concept that was coined in parallel with the three concepts above -- which is not exclusively related to use cases -- but more validation in general, is the stack-replay concept. It is neither implemented, nor designed further than this concept.
\begin{figure}[ht]
\centering
\includegraphics[width=0.7\textwidth]{\imgdir event-stack-to-state-machine}
\caption{Concept; validate event stack using life-cycle state machines.}
\label{fig:event-stack-to-state-machine}
\end{figure}
%\section{Object tracking} NOTE: Maybe something about object lifecycles (and statemachines for them) here.
Example; in the use case it is stated that a call is hung up and a callee awaits this event. The lifeline of the call is however not tracked and to be able to properly assert the true state of this, the code macro needs to into account this lifeline and reflect on which assertions hold for every stakeholder that has knowledge of the call. % TODO: Elaborate the example and explain that a phone call is a good example because it has an A and B-leg and potentially a system that tracks its state.

\chapter{Additional concepts}
This section contains concepts that was worked on, during the preparation of this thesis, but did not find its way into the main application or design. The concepts are very rough, but left in for reference.
\section{Event stack validation concept}
\label{appendix:event-stack-validation}
Event stack validation is a concept that was coined in parallel with state concept in chapter \ref{ch:design}. It is not exclusively related to use cases -- but more validation in general. It is a stack-replay concept. It is neither implemented, nor designed further than this concept.\medskip

\begin{figure}[ht]
\centering
\includegraphics[width=0.7\textwidth]{\imgdir event-stack-to-state-machine}
\caption{Concept; validate event stack using life-cycle state machines.}
\label{fig:event-stack-to-state-machine}
\end{figure}

\noindent
Example; in the use case it is stated that a call is hung up and a callee awaits this event. The lifeline of the call is however not tracked and to be able to properly assert the true state of this, the code macro needs to into account this lifeline and reflect on which assertions hold for every stakeholder that has knowledge of the call.\medskip

\noindent
If we are to define for the tests system what valid transitions are, by creating state machines, and then store object state transitions, we can assert that no objects within the system will break causality -- at least in the situations elaborated in the use cases.\medskip

\section{Further analysis of use cases}
When we have our use case represented as a graph, we can perform these additional analysis':

\begin{description}
  
  \item[Skipping actions may be prohibited:] Should it be possible to jump ahead in the use case?

  \item[Primary actor must participate:] The primary actor is important, as this is the stakeholder that defines the perspective and scope of the test. The primary is the person that starts the use case via an active action, or receives a start signal from another actor -- the system for instance. The primary actor must also be part of the main scenario, and an analysis error should occur if this is not the case.
\end{description}

\noindent Additionally, if we assumed that we had a complete semantic model of a use case, as in a full linguistically analyzed sentence with subject, verb object \emph{and} mappings to the implemented system, we could do additional checks for errors in the use case, via the tool. We would also be able to extract capabilities of an actor easily by just getting a list of verbs where the actor acted as subject.



\chapter{Handbook}

\chapter{Additional diagrams}

\begin{figure}[!htbp]
  \centering
  \includegraphics[scale=0.55]{\imgdir concept2_example_object}
  \caption{Object diagram showing the use case as mapped test, using the meta model from concept 2 (see chapter \ref{chap:conceptual_design})}
  \label{fig:concept2_example_object}
\end{figure}

\chapter{Use cases}
\label{appendix:use-cases}
\begin{usecase}
\addtitle{Use Case 1}{Transfer to contact} 

%Scope: the system under design
\addfield{Scope:}{System-wide}

%Description: A brief description of the use case
\addfield{Description:}{The receptionist actor must be able to transfer an active call to a chosen -- dialed -- contact associated with the currently active reception}

%Level: "user-goal" or "subfunction"
\addfield{Level:}{User-goal}

%Primary Actor: Calls on the system to deliver its services.
\addfield{Primary Actor:}{Receptionist actor}

%Stakeholders and Interests: Who cares about this use case and what do they want?
\additemizedfield{Stakeholders and Interests:}{
	\item Receptionist: Wants to process the call with regards to the caller's wishes
	\item Caller: Wants to reach a specific contact
}

%Preconditions: What must be true on start and worth telling the reader?
\additemizedfield{Preconditions:}{
      \item Receptionist has picked up incoming call from caller
      \item Receptionist has parked incoming call
}
%when multiple
%\additemizedfield{Preconditions:}{} 

%Postconditions: What must be true on successful completion and worth telling the reader
\additemizedfield{Postconditions:}{
      \item Caller and contact's phones are connected
      \item Receptionist is no longer in call and ready for next call      
}
%when multiple
%\additemizedfield{Preconditions:}{}

%Main Success Scenario: A typical, unconditional happy path scenario of success.
\addscenario{Main Success Scenario:}{
      \item Receptionist dials a number of the selected contact
      \item The contact accepts the call (picks up)
      \item Receptionist has a dialogue with contact
      \item Receptionist transfers contact to caller
      \item Receptionist marks his/her state as idle.
}

%Extensions: Alternate scenarios of success or failure.
\addscenario{Extensions:}{
	\item[2.a] Contact cannot be reached
		\begin{enumerate}
		\item[1.] Receptionist tries alternate contact
		\item[2.] Receptionist returns to step 1
		\end{enumerate}
	\item[3.a] Contact declines transfer
		\begin{enumerate}
		\item[1.] Receptionist hangs up contact
		\item[2.] Receptionist picks up caller
		\item[3.] Receptionist offers caller to leave a message
		\item[4a.] Caller wishes to leave a message
		\begin{enumerate}
			\item[1.] Receptionist types in message and caller information
			\item[2.] Receptionist saves message
		\end{enumerate}
		\item[4b.] Caller does not wish to leave a message
		\item[5] Receptionist ends call with caller
		\item[6] Receptionist returns to step 6
		\end{enumerate}
}
\end{usecase}

\begin{usecase}
\addtitle{Use Case 2}{Send Message to contact} 

%Scope: the system under design
\addfield{Scope:}{System-wide}

%Description: A brief description of the use case
\addfield{Description:}{A receptionist must be able to send a message -- via a distribution list -- to a contact, typically containing information received verbally via a call. An example use case, from the receptionist actor point of view is outlined below.}

%Level: "user-goal" or "subfunction"
\addfield{Level:}{User-goal}

%Primary Actor: Calls on the system to deliver its services.
\addfield{Primary Actor:}{Receptionist actor}


\additemizedfield{Preconditions:}{
      \item Receptionist have selected a contact who will serve as message recipient
} 

%Postconditions: What must be true on successful completion and worth telling the reader
\additemizedfield{Postconditions:}{
      \item Message is stored and ready for dispatching
      \item Receptionist is idle
}

%Main Success Scenario: A typical, unconditional happy path scenario of success.
\addscenario{Main Success Scenario:}{
      \item Receptionist types in message
      \item Receptionist sends the message via the system
      \item Receptionist marks his/her state as idle.
}

\end{usecase}



%\chapter{Tests}
%TODO table of tests. Expected values and actual values.

\chapter{Protocol specification}
%TODO taken at 2015-06-29
Any interface will return 500 error codes on server errors, and 400 bad request upon bad client requests.
\begin{verbatim}
[FINE] tcctool.router: GET	->	/actor
[FINE] tcctool.router: GET	->	/dummy
[FINE] tcctool.router: GET	->	/actor/{id}
[FINE] tcctool.router: PUT	->	/actor/{id}
[FINE] tcctool.router: POST	->	/actor/{name}
[FINE] tcctool.router: DELETE	->	/actor/{name}
[FINE] tcctool.router: GET	->	/concept
[FINE] tcctool.router: GET	->	/concept/{id}
[FINE] tcctool.router: PUT	->	/concept/{id}
[FINE] tcctool.router: POST	->	/concept
[FINE] tcctool.router: DELETE	->	/concept/{id}
[FINE] tcctool.router: GET	->	/template
[FINE] tcctool.router: POST	->	/template
[FINE] tcctool.router: GET	->	/template/{tplid}
[FINE] tcctool.router: PUT	->	/template{tplid}
[FINE] tcctool.router: GET	->	/usecase
[FINE] tcctool.router: GET	->	/usecase/{id}
[FINE] tcctool.router: PUT	->	/usecase/{id}
[FINE] tcctool.router: POST	->	/usecase/{id}
[FINE] tcctool.router: DELETE	->	/usecase/{id}
[FINE] tcctool.router: POST	->	/usecase/{id}/testsfromtemplate/{tplid}/
[FINE] tcctool.router: POST	->	/test/{tid}/analyze
[FINE] tcctool.router: GET	->	/configuration
[FINE] tcctool.router: PUT	->	/configuration
\end{verbatim}

\chapter{Database schema}
\lstinputlisting[language=SQL, caption=Database schema for the test generation tool ]{../tool/service/db/schema.sql}