\chapter{Additional concepts}
This section contains concepts that was worked on, during the preparation of this thesis, but did not find its way into the main application or design. The concepts are very rough, but left in for reference.
\section{Event stack validation concept}
\label{appendix:event-stack-validation}
Event stack validation is a concept that was coined in parallel with state concept in chapter \ref{ch:design}. It is not exclusively related to use cases -- but more validation in general. It is a stack-replay concept. It is neither implemented, nor designed further than this concept.\medskip

\begin{figure}[ht]
\centering
\includegraphics[width=0.7\textwidth]{\imgdir event-stack-to-state-machine}
\caption{Concept; validate event stack using life-cycle state machines.}
\label{fig:event-stack-to-state-machine}
\end{figure}

\noindent
Example; in the use case it is stated that a call is hung up and a callee awaits this event. The lifeline of the call is however not tracked and to be able to properly assert the true state of this, the code macro needs to into account this lifeline and reflect on which assertions hold for every stakeholder that has knowledge of the call.\medskip

\noindent
If we are to define for the tests system what valid transitions are, by creating state machines, and then store object state transitions, we can assert that no objects within the system will break causality -- at least in the situations elaborated in the use cases.\medskip

\section{Further analysis of use cases}
When we have our use case represented as a graph, we can perform these additional analysis':

\begin{description}
  
  \item[Skipping actions may be prohibited:] Should it be possible to jump ahead in the use case?

  \item[Primary actor must participate:] The primary actor is important, as this is the stakeholder that defines the perspective and scope of the test. The primary is the person that starts the use case via an active action, or receives a start signal from another actor -- the system for instance. The primary actor must also be part of the main scenario, and an analysis error should occur if this is not the case.
\end{description}

\noindent Additionally, if we assumed that we had a complete semantic model of a use case, as in a full linguistically analyzed sentence with subject, verb object \emph{and} mappings to the implemented system, we could do additional checks for errors in the use case, via the tool. We would also be able to extract capabilities of an actor easily by just getting a list of verbs where the actor acted as subject.