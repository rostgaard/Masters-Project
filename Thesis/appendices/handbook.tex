\chapter{Handbook}
\label{appendix:handbook}
This appendix is a handbook of the use-case writing and test generation tool implemented in the context of this thesis. The handbook presents the software through a number of screenshots, and an explanation on how to write a use case, and generate a test for it\medskip

\begin{figure}[!htbp]
  \centering
  \includegraphics[scale=0.45]{\imgdir screenshot-actors}
  \caption{The actors panel of the application}
\label{fig:screenshot-actors}
\end{figure}
\begin{figure}[!htbp]
  \centering
  \includegraphics[scale=0.45]{\imgdir screenshot-concepts}
  \caption{The concepts panel of the application}
\label{fig:screenshot-concepts}
\end{figure}

\noindent Figure \ref{fig:screenshot-actors} shows the user interface, currently navigated to the ``Actors'' panel. The left side of the interface show the navigation options. The ``Actors'' panel has the possibility to a new actor definition. In order to do this, the actor must be named and given a role. The description is optional, and the actor definition will be created, once the ``Create'' button is pressed. An existing definitions may be removed from the list below by pressing the ``Remove'' button to the left of it. \medskip

\begin{figure}[!htbp]
  \centering
  \includegraphics[scale=0.45]{\imgdir screenshot-use-cases}
  \caption{The use cases panel of the application}
\label{fig:screenshot-use-cases}
\end{figure}

\noindent The next navigation option is the ``Concepts'' panel. This panel is equivalent to the ``Actors'' panel in functionality, and we refer to this section for usage information. A screenshot of the ``Concepts'' panel is shown it figure \ref{fig:screenshot-concepts}.\medskip

\noindent The ``Use cases'' panels (figure \ref{fig:screenshot-use-cases}) contains a number of different components. The topmost one is the ``Current use case'' selector, which is a drop-down that will change the current use case. The remainder of the panel is split into an ``Edit'' region, and a ``Preview'' region. The ``Edit'' region allows editing of a use case, such as name or description change, or modification of the scenario, pre- and postconditions. Each step may be modified by the navigation buttons to the left of the list item. The up arrow moves the step up in the list, the down arrow moves it down. The ``x'' removes the step from the use case. New steps can be added via the associated input field and add button.\medskip


\noindent 

\begin{figure}[!htbp]
  \centering
  \includegraphics[scale=0.45]{\imgdir screenshot-generate-tests}
  \caption{The test generation panel of the application}
\label{fig:screenshot-screenshot-generate-tests}
\end{figure}

\begin{figure}[!htbp]
  \centering
  \includegraphics[scale=0.45]{\imgdir screenshot-templates}
  \caption{The test templates panel of the application}
\label{fig:screenshot-templates}
\end{figure}

\begin{figure}[!htbp]
  \centering
  \includegraphics[scale=0.45]{\imgdir screenshot-configuration}
  \caption{The configuration panel of the application}
\label{fig:screenshot-configuration}
\end{figure}