\chapter{Glossary}
\label{appendix:glossary}
This appendix contains a list of terms introduced or used in this thesis.
\begin{description}

  \item[Domain framework:] A hand-coded set of source code files that exposes from, and shares interfaces, and model classes, with the system under test and test support library.

  \item[Test support framework:] A hand-coded set of source code files that provides interfacing between the sources of application under development, and the tests files that are generated. In this thesis, it builds upon an existing test library designed for unit testing. It provides the needed ``setup'', ``teardown'', grouping and tests runners that catch and report unhandled exceptions.

  \item[CRUD:] Abbreviation of \textbf{C}reate, \textbf{R}ead \textbf{U}pdate and \textbf{D}elete is an acronym denoting the four primitives operation of persistent data storage. An interface for persistent storage may expose all or a subset of these primitives.

  \item[Requirement analysis:] Requirement analysis, for the purpose of this project, meant as a concept that uses some the information stored in the requirements to check it for validity, ambiguity and provide alternative representations. An example of an alternative representation is that a set of use cases becomes a use a use case diagram, just on a different abstraction level.
  
  \item[System under development:] The software system that is being developed, and serves as the system under test in the context of this thesis.
  
  \item[System under test:] The software system that is being tested against. Equivalent to ``system under development'' in the context of this thesis.
  
\end{description}
