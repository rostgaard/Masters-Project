%!TEX root = ../Thesis.tex
\chapter{Summary}
The world of software has over the years been tainted with stories about failed projects. Almost everyone has his or her on story about a given software system, that did not solve the task, which it was designed to. This indicates; either a mismatch between requirements and solution, or simply erroneous requirement specifications. In the last decades more emphasis have been put on the agile methods, where smaller time-boxed iterations intricately contains every development phase; design, implementation, documentation and validation. Requirement refinement and elicitation may, however, be left dead-in-the-water and remain a \emph{de facto} waterfall phase with neither feed-back from implementation and validation, nor feed-forward to validation.\bigskip

\noindent Adding enough structure, and/or formalism to be able to automatically generate artifacts -- such as documentation, diagrams and even tests -- from requirements is a careful balance. This balance is between strict rules for writing requirements, and usability. The balance is also exactly what is sought by -- and is thus also the topic of -- this thesis
\newpage
\begin{fquote}[\small -- Johann Wolfgang von Goethe][The Sorrows of Young Werthe][1774]Misunderstandings and neglect occasion more mischief in the world than even malice and wickedness. At all events, the two latter are of less frequent occurrence.
\end{fquote}